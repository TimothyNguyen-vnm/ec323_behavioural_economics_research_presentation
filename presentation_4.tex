\documentclass{beamer}
\usepackage{amsmath}
\usepackage{graphicx} % Using this package to include images

%----------------------------------------------------------------------------------------
%	THEME AND APPEARANCE
%----------------------------------------------------------------------------------------
\usetheme{Madrid}
\usecolortheme{default}
\setbeamertemplate{navigation symbols}{}
\setbeamertemplate{footline}[frame number]

%----------------------------------------------------------------------------------------
%	TITLE PAGE
%----------------------------------------------------------------------------------------
\title[Altruistic Punishment]{The Price of Justice: Altruistic Punishment and Human Cooperation}
\subtitle{Based on Fehr \& Gächter (Nature, 2002)}
\author{Timothy Nguyen}
\institute{EC323: Behavioral Economics \\ Professor Jawwad Noor}
\date{August 6, 2025}

\begin{document}

\begin{frame}
\maketitle
\end{frame}

%----------------------------------------------------------------------------------------
%	INTRODUCTION
%----------------------------------------------------------------------------------------
\begin{frame}{The Puzzle of Cooperation}
    \begin{itemize}
        \item<1-> Humans often help strangers, even at a personal cost.
        \pause
        \item<2-> This is a puzzle because standard economic models assume people are purely self-interested.
        \pause
        \item<3-> So, why do we punish "free-riders" when we get nothing out of it?
        \pause
        \item<4-> This is called \textbf{altruistic punishment}.
    \end{itemize}
\end{frame}

%----------------------------------------------------------------------------------------
%	CONNECTION TO LECTURE
%----------------------------------------------------------------------------------------
\begin{frame}{Connecting to Our Class: Social Preferences}
    This experiment challenges the basic assumption of pure self-interest.
    \begin{itemize}
        \item<1-> In our lectures, we've discussed how people are not always selfish. They have \textbf{social preferences}.
        \pause
        \item<2-> People care about fairness and what others get. This is sometimes called \textbf{inequity aversion}.
        \pause
        \item<3-> Altruistic punishment is a powerful example of this: people are willing to \textit{pay money} just to reduce the payoff of someone they think is unfair.
        \pause
        \item<4-> This behavior shows that a sense of justice can be a stronger motivator than money.
    \end{itemize}
\end{frame}

%----------------------------------------------------------------------------------------
%	EXPERIMENT DESIGN
%----------------------------------------------------------------------------------------
\begin{frame}{The Experiment: A "Public Goods" Game}
    \textbf{The Setup:}
    \begin{itemize}
        \item Anonymous groups of 4.
        \item Each person gets 20 Money Units (MUs).
        \item You can invest in a group project.
    \end{itemize}
    \pause
    \textbf{The Dilemma:}
    \begin{itemize}
        \item For every 1 MU you invest, everyone in the group gets 0.4 MUs back.
        \item Your best strategy is to be a \textbf{free-rider}: invest nothing and just collect the rewards from others' investments.
        \item But if everyone free-rides, everyone is worse off!
    \end{itemize}
\end{frame}

%----------------------------------------------------------------------------------------
%	CONDITIONS
%----------------------------------------------------------------------------------------
\begin{frame}{The Two Conditions}
    To see if punishment works, subjects were split into two scenarios.
    \begin{itemize}
        \item<1-> \textbf{Condition 1: No Punishment}
        \begin{itemize}
            \item You play the game, see what others invested, and that's it. You can't do anything about it.
        \end{itemize}
        \pause
        \item<2-> \textbf{Condition 2: Punishment Allowed}
        \begin{itemize}
            \item After investing, you have the option to punish others.
            \item \textbf{The Cost of Justice:} You can spend 1 MU to make someone else lose 3 MUs.
            \item This is "altruistic" because it costs you money and gives you no direct financial reward.
        \end{itemize}
    \end{itemize}
\end{frame}

%----------------------------------------------------------------------------------------
%	RESULTS 1
%----------------------------------------------------------------------------------------
\begin{frame}{Result 1: Punishment Boosts Cooperation}
    \begin{columns}[c]
        \begin{column}{.5\textwidth}
            The effect of punishment was huge.
            \begin{itemize}
                \item<1-> \textbf{Without Punishment:} Cooperation collapsed. People quickly learned to free-ride.
                \pause
                \item<2-> \textbf{With Punishment:} Cooperation was high and even \textit{increased} over time.
                \pause
                \item<3-> The graph shows this clearly. The top line in (a) and the bottom-right line in (b) show cooperation thriving when punishment is possible.
            \end{itemize}
        \end{column}
        \begin{column}{.5\textwidth}
            \begin{center}
            % This is the line graph image you provided
            \includegraphics[width=\textwidth]{figure2.png}
            \end{center}
        \end{column}
    \end{columns}
\end{frame}

%----------------------------------------------------------------------------------------
%	RESULTS 2
%----------------------------------------------------------------------------------------
\begin{frame}{Result 2: Punishment Is Targeted}
    \begin{columns}[c]
        \begin{column}{.5\textwidth}
            Punishment wasn't random.
            \begin{itemize}
                \item<1-> People consistently punished free-riders (those who contributed less than average).
                \pause
                \item<2-> The more someone deviated from the average, the more heavily they were punished.
                \pause
                \item<3-> This graph shows that the biggest free-riders (far left, "-20 to -14") received the most punishment.
                \pause
                \item<4-> This suggests punishment is driven by a reaction to unfairness.
            \end{itemize}
        \end{column}
        \begin{column}{.5\textwidth}
            \begin{center}
            % This is the bar chart image you provided
            \includegraphics[width=\textwidth]{figure1.png}
            \end{center}
        \end{column}
    \end{columns}
\end{frame}

%----------------------------------------------------------------------------------------
%	CONCLUSION
%----------------------------------------------------------------------------------------
\begin{frame}{Conclusions \& Take-Aways}
    \begin{itemize}
        \item<1-> Altruistic punishment is a powerful tool that helps enforce cooperation.
        \pause
        \item<2-> This behavior is likely driven by our emotional hatred of unfairness, not a cold calculation of benefits.
        \pause
        \item<3-> It shows that "social preferences" can sometimes be more important than pure self-interest in making economic decisions.
        \pause
        \item<4-> To understand cooperation, we need to understand our willingness to enforce fairness, even at a personal cost.
    \end{itemize}
\end{frame}

%----------------------------------------------------------------------------------------
%	DISCUSSION
%----------------------------------------------------------------------------------------
\begin{frame}{Personal Comments \& Discussion}
    \textbf{My Reactions:}
    \begin{itemize}
        \item It's fascinating that a "negative" emotion like anger can actually lead to better outcomes for the group.
        \item The simple design of the experiment makes the results very powerful and easy to understand.
        \item This research has huge real-world implications for everything from managing climate change to building online communities.
    \end{itemize}
    \vspace{1cm}
    \textbf{Questions for Discussion:}
    \begin{itemize}
        \item Can you think of real-world examples of altruistic punishment?
        \item Is this behavior truly "altruistic" if punishing others makes us feel good?
        \item When can this desire to punish unfairness go wrong?
    \end{itemize}
\end{frame}

%----------------------------------------------------------------------------------------
%	Q&A
%----------------------------------------------------------------------------------------
\begin{frame}
    \begin{center}
        \Huge{\textbf{Thank You}}
        \vspace{2cm}
        \Large{Questions?}
    \end{center}
\end{frame}

\end{document}